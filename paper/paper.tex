\documentclass[11 pt]{extarticle}

\usepackage[margin=1in]{geometry}
\usepackage{amsmath,amsthm,amssymb}
\usepackage[shortlabels, inline]{enumitem}
\usepackage{mleftright}
\usepackage{array}
\usepackage{gensymb}
\usepackage[english]{babel}
\usepackage{setspace}
\usepackage{pgfplots}
\usepackage{booktabs}
\usepackage[numbers]{natbib}
\usepackage{todonotes}
\newcommand\mdoubleplus{\mathbin{+\mkern-10mu+}}
\usepackage{tikz-cd}
% http://tex.stackexchange.com/q/169557/5764
\usepackage{mathtools}
\DeclarePairedDelimiter{\norm}{\lVert}{\rVert}
\newcommand{\vectorproj}[2][]{\textit{proj}_{\vect{#1}}\vect{#2}}
\newcommand{\vect}{\mathbf}
\pgfplotsset{width=10cm,compat=1.9}


\newcommand{\N}{\mathbb{N}}
\newcommand{\Z}{\mathbb{Z}}
\newcommand{\R}{\mathbb{R}}
\newcommand{\C}{\mathbb{C}}
\newcommand{\Q}{\mathbb{Q}}
\newcommand{\Mat}{\text{Mat}}
\newcommand{\sgn}{\text{sgn}}
\DeclareMathOperator{\lcm}{lcm}
\DeclareMathOperator{\img}{im}

\let\hom\relax% Set equal to \relax so that LaTeX thinks it's not defined
\DeclareMathOperator{\hom}{Hom}
\newcommand{\Char}{\text{char}}
\newcommand{\defeq}{\overset{\text{\tiny def}}{=}}

\theoremstyle{remark}
\newtheorem{lemma}{Lemma}


\newenvironment{amatrix}[1]{%
  \left[\begin{array}{@{}*{#1}{c}|c@{}}
          }{%
        \end{array}\right]
    }



    \renewcommand\qedsymbol{$\blacksquare$}


    \newenvironment{theorem}[2][Theorem]{\begin{trivlist}
      \item[\hskip \labelsep {\bfseries #1}\hskip \labelsep {\bfseries #2.}]}{\end{trivlist}}
    \newenvironment{theorem*}[1][Theorem:]{\begin{trivlist}
      \item[\hskip \labelsep {\bfseries #1}]}{\end{trivlist}}
    \newenvironment{exercise}[2][Exercise]{\begin{trivlist}
      \item[\hskip \labelsep {\bfseries #1}\hskip \labelsep {\bfseries #2.}]}{\end{trivlist}}
    \newenvironment{reflection}[2][Reflection]{\begin{trivlist}
      \item[\hskip \labelsep {\bfseries #1}\hskip \labelsep {\bfseries #2.}]}{\end{trivlist}}
    \newenvironment{proposition}[2][Proposition]{\begin{trivlist}
      \item[\hskip \labelsep {\bfseries #1}\hskip \labelsep {\bfseries #2.}]}{\end{trivlist}}
    \newenvironment{corollary}[2][Corollary]{\begin{trivlist}
      \item[\hskip \labelsep {\bfseries #1}\hskip \labelsep {\bfseries #2.}]}{\end{trivlist}}

    \begin{document}
    \onehalfspacing

    \title{Zero-Knowledge Credentials for Smart Contracts}
    \author{Lucas Switzer, Tjaden Hess}

    \maketitle

    \begin{abstract}
      Public blockchains present unique opportunities for the implementation of
      autonomous and trustless systems, but suffer from trade-offs between
      privacy and expressivity. In this paper we present an implementation of a
      zkSNARK-based
      anonymous credential scheme for the Ethereum blockchain and give
      benchmarks for usage costs. We present as well an example application.
    \end{abstract}


    \section{Introduction}
    While blockchains have found use cases in publicly accessible distributed
    systems, they pose a challenge in that due to their public nature it is
    currently impossible to attest to aspects of one's identity without some
    trusted credential issuer.

    \subsection{Prior Work}
    A generalized scheme was proposed by \citet{garmanDecentralizedAnonymousCredentials2013}

    \section{Properties and Features}

    \section{Implementation}

    \section{Benchmarks}

    \section{Example Application}


    \section{Future Work}

    \section{Conclusion}

    \bibliographystyle{plainnat}
    \bibliography{Identity}
  \end{document}
  
